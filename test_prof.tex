\documentclass[a4paper, 12pt]{article}
\usepackage{qrcode} % Pour générer des QR codes
\usepackage{amsmath} % Pour les formules mathématiques
\usepackage{graphicx} % Pour inclure des graphiques (si nécessaire)
\usepackage{xcolor}  % Pour utiliser des couleurs
\usepackage{verbatim} % Pour utiliser \comment et \endcomment

% Version Prof
\newif\ifProf
\Proftrue % Mettre \Proftrue pour activer les corrections

% Environnement pour les exercices
\newenvironment{Exercice}[1][]{
    \section*{Exercice : #1}
}{}

% Environnement pour les corrections conditionnelles
\ifProf
    % Si Prof est True, on affiche le contenu en bleu
    \newenvironment{Correction}[1][]{
        \subsection*{\color{blue}Correction : #1}
        \color{blue} % Définit la couleur bleue pour tout le contenu de l'environnement
    }{}
\else
    % Si Prof est False, on définit un environnement qui ignore complètement le contenu
    \newenvironment{Correction}[1][]{\comment}{\endcomment}
\fi


\begin{document}

% Exemple d'exercice
\begin{Exercice}[Calcul de la dérivée d'une fonction]
    % Données de l'exercice
    Calculer la dérivée de la fonction $f(x) = 2x^3 + 4x^2 - 1$ en fonction de $x$.
    % Correction de l'exercice
    \begin{Correction}[Dérivation de $f(x) = 2x^3 + 4x^2 - 1$]
        La dérivée de $f(x) = 2x^3 + 4x^2 - 1$ est $f'(x) = 6x^2 + 8x$.
    \end{Correction}
\end{Exercice}


\begin{Exercice}[Calcul de la dérivée d'une fonction]
    % Données de l'exercice
    Calculer la dérivée de la fonction $f(x) = 12x^3 + 4x^2 - 1$ en fonction de $x$.
    % Contenu de la correction
    \begin{Correction}[Dérivation de $f(x) = 12x^3 + 4x^2 - 1$]

        La dérivée de $f(x) = 12x^3 + 4x^2 - 1$ est $f'(x) = 36x^2 + 8x$.
    \end{Correction}
\end{Exercice}

\end{document}