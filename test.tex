\documentclass[a4paper, 12pt]{article}
\usepackage{qrcode} % Pour générer des QR codes
\usepackage{amsmath} % Pour les formules mathématiques
\usepackage{graphicx} % Pour inclure des graphiques (si nécessaire)
\usepackage{xcolor}  % Pour utiliser des couleurs si nécessaire

% Environnement pour les exercices
\newenvironment{Exercice}[1][]{
    \section*{Exercice : #1}
}{}

% Environnement pour les corrections
\newenvironment{Correction}[1]{
    \begin{tabular}{|c|c|}
        \hline
    \qrcode{https://example.com/#1} &
    \textbf{Correction} \\ \hline
    \end{tabular}\par
}{} 

\begin{document}

% Exemple d'exercice
\begin{Exercice}[Calcul de la dérivée d'une fonction]
    % Données de l'exercice
    Calculer la dérivée de la fonction $f(x) = 2x^3 + 4x^2 - 1$ en fonction de $x$.

    \begin{Correction}{exercice_1_correction.pdf}
        % Contenu de la correction
        La dérivée de $f(x) = 2x^3 + 4x^2 - 1$ est $f'(x) = 6x^2 + 8x$.
    \end{Correction}
\end{Exercice}


\begin{Exercice}[Calcul de la dérivée d'une fonction]
    % Données de l'exercice
    Calculer la dérivée de la fonction $f(x) = 12x^3 + 4x^2 - 1$ en fonction de $x$.

    \begin{Correction}{exercice_2_correction.pdf}
        % Contenu de la correction
        La dérivée de $f(x) = 12x^3 + 4x^2 - 1$ est $f'(x) = 36x^2 + 8x$.
    \end{Correction}
\end{Exercice}

\end{document}